\documentclass{article}
\usepackage{graphicx}
\usepackage{multirow}
\usepackage{caption}
\usepackage[utf8]{inputenc}
  \usepackage[
    backend=biber,
    style=ieee,
  ]{biblatex}
 


\title{CS6000 - Fifth Assignment}
\author{Marc Moreno Lopez}
\date{October 1st 2018}

\begin{document}

\maketitle

\section{Report}

%Describe your learning/process so far on the survey paper.  I expect this to be short. 

During this week I have been reading some of the papers more in depth, skimming through them (5-10 min each), to extract more information. This way, it will be easier to detect the common features that the papers that are under the same category share between them. This will also help when deciding which figures to include in the paper. I won’t skim through all of them, but I do want to get the main ideas behind the ones that I think that are the most important ones (most cited ones, papers that I have like the most).

After doing the categorization of papers that we did last week, I have been thinking a way in which we could break up or organize the papers that are under the miscellaneous category. They are interesting papers, but they don’t share many features between them. 

One thing that worries me a little are the figures that we are going to use for the paper. So far, I’m struggling to see what we should put in figures. Figures are an important part of the paper, since when people do the first read, they tend to fixate on figures. I would like to come up with figures that are significant for the paper and they catch the eye of the person reading it.

Inside my repo I have cloned the overleaf project. In this folder you can find all the files for the tex file.




\end{document}
